
\documentclass{article}

\usepackage[spanish]{babel}
\usepackage{amsmath}
\usepackage[utf8]{inputenc}

\title{El método símplex}
\author{Christian G.}

\begin{document}
\maketitle
\section{Introducción}

El método simplex es un algoritmo para resolver problemas de programación lineal. Fue inventado por George Dantzing en el año 1947.

\section{Ejemplo}
\label{sec:ejemplo}

Ilustraremos la aplicación del método simplex con un ejemplo.


Resuelve el siguiente problema utilizando el método símplex:

\begin{equation*}
\begin{aligned}
\text{Minimizar} \quad & x_{1}-x_{2}\\
\text{sujeto a} \quad &
  \begin{aligned}
   3x_{1}-x_{2} & \geq 3\\
   x_{1}+x_{2} &\geq -8\\
   x_{1},x_{2} &\geq 0
  \end{aligned}
\end{aligned}
\end{equation*}\\


Para aplicar el método simplex, debemos tener el problema expresado, en primera instáncia en forma ''estándar'', el cual nos pide maximizar en vez de minimizar, y las variables deben de estar del lado izquierdo de la desigualdad, por tanto, para corregir esto:\\

Minimizar $x_{1}-x_{2}$ es equivalente a Maximizar $-x_{1}+x_{2}$, es decir, se multiplica por $-1$ la expresión a minimizar.\\

Tener $3x_{1}-x_{2} \geq 3$ es equivalente a tener $-3x_{1}+x_{2} \leq -3$.\\


Analogamente $x_{1}+x_{2} \geq -8$ es equivalente a tener $-x_{1}-x_{2} \leq 8$\\

Por lo cual, nuestro problema en su forma estándar es:\\


\begin{equation*}
\begin{aligned}
  \text{Maximizar} \quad & -x_{1}+x_{2}\\
\text{sujeto a} \quad &
  \begin{aligned}
   -3x_{1}+x_{2} & \leq -3\\
   -x_{1}-x_{2} &\leq 8\\
   x_{1},x_{2} &\geq 0
  \end{aligned}
\end{aligned}
\end{equation*}



\end{document}



