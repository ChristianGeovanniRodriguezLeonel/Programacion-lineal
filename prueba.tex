\documentclass{article}

\title{Apuntes de programaci\'on lineal}

\author{Christian G. Rodr\'iguez Leonel}

\usepackage[utf8]{inputenc}

\usepackage{amsmath}

\usepackage[spanish]{babel}

\begin{document}

\maketitle

\tableofcontents

\section{Introducción}

\label{sec:introduccion}

\subsection{Forma estándar}





La forma estándar de un problema de progamación lineal es:

Dados una matriz $A$ y vectores $b,c$, maximizar $c^{T}x$ sujeto a $Ax\leq b$.

Maximizar: $C_{1}X_{1}+ C_{2}X_{}2+ ... +C_{n}X_{n}$


\subsection{Forma simplex}


La forma ``Simplex'' de un problema de programación lineal es:

Dados una matriz $A$ y vectores $b,c$, maximizar $c^{T}x$ sujeto a $Ax=b$.

\begin{tabular}{c|c|c}
  & A & B \\
  \hline
  
  Maquina 1 & 1 & 2 \\
  Maquina 2 & 1 & 1

\end{tabular}

\begin{equation}
  \label{eq:1}
 A=
  \begin{pmatrix}
    0 & 1 & 55\\
    5 & 9 & 1
  \end{pmatrix}
  \begin{pmatrix}
    1 & 0\\
    2 & 99 \\
    8 & 0
  \end{pmatrix}
\end{equation}
\end{document}



